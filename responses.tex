\newcommand{\RR}{\ensuremath{{\mathcal R}}}
\newcommand{\Rhat}{\ensuremath{{\hat\RR}}}
\newcommand{\rr}{\ensuremath{{r}}}

\newcommand{\comment}{RENEW the comment command}
\renewcommand{\comment}[3]{}
\renewcommand{\comment}[3]{\textcolor{#1}{\textbf{[#2: }\textit{#3}\textbf{]}}}
\newcommand{\jd}[1]{\comment{cyan}{JD}{#1}}
\newcommand{\swp}[1]{\comment{magenta}{SWP}{#1}}

\hypertarget{referee-1}{%
\subsection{Referee: 1}\label{referee-1}}

\begin{quote}\sl
One additional reference that might be worth mentioning is O. Diekmann,
H. Heesterbeek, and H. Metz,The legacy of Kermack and McKendrick, in
Epidemic Models, Their Structure and Relation to Data (D. Mollinson,
ed.) (1995), pp.95-115.
\end{quote}

Thank you. We now cite this article as an early mentioner of the
little-r threshold.

\hypertarget{referee-2}{%
\subsection{Referee: 2}\label{referee-2}}

\begin{quote}\sl
In this work the authors propose that ``strength'' and ``speed'' are
complementary frameworks for understanding interventions in infectious
disease models. This is an interesting and appealing idea based on
simple re-arrangements of well-known equations for exponentially growing
epidemics. As the authors themselves note, comparing growth rates r
rather than reproductive numbers R is already well known in ecology.
\end{quote}

\begin{quote}\sl
In my view this paper could bring these ideas to light in infectious
disease more thoroughly and compellingly with an expanded application
section, a section on the relationship to data and/or estimation from
data, improved and expanded section on interventions and the distinct
views provided by (and usefulness of) the two distinct frameworks,
and/or with some extension of the results to population dynamics that
differ from constant rate exponential growth or decay. As it stands it
feels quite minimal, though I appreciate the appeal of the conceptual
point.
\end{quote}

Thank you for this feedback. We agree that these ideas are similar to
ideas that have been discussed in the ecological context, but also feel
that this is a perspective sorely lacking in the disease-modeling world.

We have expanded the application section with examples from COVID-19. To
our mind, these examples further underline the absence of \rr-based
thinking in the disease-modeling world. We also address more explicitly
the link between the parameters of the initial phase and the prospects
for long-term control. We feel that issues dealing directly with
non-exponential disease dynamics are complicated enough for another
paper.
\swp{Seems like they're repeated. Delete the first one?}
We have expanded the
application section with examples from COVID-19, providing a qualitative
understanding on how uncertainties in the amount of presymptomatic
transmission can affect conclusions about the efficacy of intervention.
In the beginning of the current pandemic, there was much discussion on
how presymptomatic transmission could make intervention more
difficult---however, some of these conclusions were based on the
assumption that \RR\ is constant, rather than \rr\ (e.g., Hellewell et
al.~2020. ``Feasibility of controlling COVID-19 outbreaks by isolation
of cases and contacts.'' \emph{ Lancet Global Health}. In reality, given
observed \rr, more presymptomatic transmission reduces \RR, meaning that
strength-based interventions will be easier than previously thought. To
our mind, these examples further underline the absence of \rr-based
thinking in the disease-modeling world. While we appreciate the
reviewer's concern that a more data-driven approach could provide
further insight, we are worried that details may take away general ideas
behind speed- and strength-based paradigms. We feel that it is important
to provide a conceptual framework at this stage to a broader audience.
We also address more explicitly the link between the parameters of the
initial phase and the prospects for long-term control. We feel that
issues dealing directly with non-exponential disease dynamics are
complicated enough for another paper.

\begin{quote}\sl
Near equation (9) the authors need to make a more explicit link with
survival analysis so that we are clear on why we multiply K by survival
functions. Survival functions multiply in the sense that if there are
two hazards at rates lambda1 and lambda2 then the survival function is
S(t) = S1(t) S2(t) . But K(tau) is proportional to the density g(tau),
and is not a survival function. What is the interpretation of the
density times the survival function? Why would the interventions not
change the density directly, so that g becomes ghat (and then this would
impact the hazard rate and survival function via the usual relationship
$S(t) = \int_t ^\infty ( ghat (x) dx )  )$ ?
\end{quote}

We have expanded (9) into two steps and attempted to explain the logic
clearly.

\begin{quote}\sl
In practice what is the need for equations (6) and (12)? Both
essentially re-write the other equations; are these averages that can be
linked to estimates from data, for example?
\end{quote}

We think that the value of (6) and (12 -- now 13) is conceptual. We have
added explanations following these derivations. In retrospect, we
believe that Ss 2.2 and 2.3 were written in too ``mathematical'' a
style: we got a bit carried away with the formal parallelism. The new
version is still short, but engages more with the biological concepts.

We note that in practice 6 and 12 are often implicit in mechanistic models. We now make this point in the MS near the end of both S2.2 and S2.3.

\begin{quote}\sl
Presumably both frameworks give a decline when the growth rate is
negative and growth when it is positive. Therefore they would never
contradict each other as to whether an intervention was sufficient or
not - is this correct? This should be made explicit.
\end{quote}

This is correct, and we have worked to be more explicit. 

\begin{quote}\sl
It would also be good to more explicitly state and discuss what the
differences are. I realize that this is in the discussion and in some
sense in Figure 4, but I found the discussion section vague,
particularly since the analysis is only relevant for the case of
exponential growth. In particular, where would the two frameworks give
different estimates of the uncertainty surrounding the benefits of an
intervention, or the comparison between two interventions?
\end{quote}

We never expect to get a different final answer from the two formalisms if they are both correctly applied;
they are two ways of doing essentially the same calculation. 
We have tried to be more clear about this as mentioned above. 
We do expect in some cases to be able to use one formalism or another to more clearly illuminate the key issues and quantities determining the detailed answer, as we try to explain in our HIV example and in our new COVID example.

Keeping the two paradigms in mind, however, can help to do the analysis correctly. 
In particular, researchers should give careful thought to which quantities to fix when considering scenarios for disease spread. 
In particular, in emerging epidemics, researchers often vary assumptions about parameters while holding \RR\ constant, even in cases where it would be more consistent with available information to hold \rr\ constant.
We now discuss this in the paper.

\begin{quote}\sl
On a minor note, since condoms are ultimately an individual choice
applied to individual (potential) transmission events, why are they a
population-level intervention whereas test and treat is deemed
individual? More details on why the two frameworks give different
intuition or analytical capacity would be good.
\end{quote}

Thank you for this. We were using individual to mean targeted at
infectious individuals, and population to mean targeted at everyone. We
have now dropped these shortcuts, and we simply explain this directly. We have also added more discussion of how frameworks differ, both in examples and in Discussion.

\begin{quote}\sl
In Figures 3 and 4, the black lines are not clear to me -- the strength
is defined as R/Rhat and the speed as rhat -r . So what is the strength
(or speed) of ``the epidemic'' (black lines) - what's Rhat and what's R?
It might be good to have some prevalence curves
\end{quote}

We define strength and speed of the epidemic in the introduction, and
compare these to strength and speed of intervention efforts. We now
reiterate these points (including in the figure captions).

\begin{quote}\sl
Minor quibble: equation (4) has L described in words as the ``average
proportional reduction'' but where the reduction is 75\%, in fact the
denominator is 4: $(1 / (1-0.75) )$ -- rephrase.
\end{quote}

Done, thank you.

