\section*{Significance statement}

Researchers studying infectious-disease outbreaks often focus primarily on the reproductive number, which measures the ``strength'' of transmission --- the number of new infections caused by an infected individual.
Less theoretical and modeling attention is paid to the exponential growth rate, which measures epidemic ``speed''. 
We argue that this difference in emphasis is largely artificial: both strength and speed have thresholds for spread and provide frameworks to compare effectiveness of intervention.
We generalize the classic, strength-based paradigm and show that there is a corresponding speed-based paradigm which can provide complementary insights.
We provide specific examples from HIV and other diseases to demonstrate the value of both perspectives.
