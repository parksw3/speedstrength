\documentclass{article}\usepackage[]{graphicx}\usepackage[]{color}

\title{Speed and strength of epidemic intervention}
\author{Jonathan Dushoff, Sang Woo Park, and?}

\usepackage{tabularx}

\usepackage{amsmath}
\usepackage{natbib}
\usepackage{hyperref}
\bibliographystyle{chicago}
\date{\today}

\usepackage{bm}

\usepackage{afterpage}
\usepackage{pdflscape}

\newcommand{\etal}{\textit{et al.}}

\newcommand{\comment}[3]{\textcolor{#1}{\textbf{[#2: }\textit{#3}\textbf{]}}}
\newcommand{\jd}[1]{\comment{cyan}{BMB}{#1}}
\newcommand{\swp}[1]{\comment{magenta}{SWP}{#1}}

\newcommand{\Rx}[1]{\ensuremath{{\mathcal R}_{#1}}} 
\newcommand{\Ro}{\Rx{0}}
\newcommand{\RR}{\ensuremath{{\mathcal R}}}
\newcommand{\Rhat}{\ensuremath{{\hat\RR}}}
\newcommand{\tsub}[2]{#1_{{\textrm{\tiny #2}}}}

\newcommand{\fref}[1]{Fig.~\ref{fig:#1}}
\begin{document}

\maketitle

\section{Introduction}

An epidemic can be characterized by its \emph{speed} and \emph{strength}.
The speed of an epidemic, often characterized by the exponential growth rate $r$, 
measures how \emph{fast} an epidemic grows initially. On the other hand, the 
strength of an epidemic, often characterized by the basic reproductive number $\Ro$,
measures how \emph{large} an epidemic will grow. Knowing the speed and strength of 
an epidemic allows us to make short- and long-term predictions.

Infectious disease research often focuses on estimating the basic reproductive
number -- defined as the average number of secondary cases caused by a primary 
case in a fully susceptible population -- because it allows us to make 
prediction about the final size of an epidemic and the difficulty of the 
associated epidemic intervention. In particuar, the required control effort 
(i.e., the proportion of the population that needs to be vaccinated in order to
prevent the disease from spreading) is often expressed as $1 - 1/\Ro$. 
\swp{Maybe a sentence about Ronald Ross?}

Like epidemic strength, epidemic speed measures how disease spreads in the population
and, therefore, can help us understand the effectiveness of a control strategy.

\section{Methods}

\subsection{Epidemic moodel}

Infection kernel, $K(\tau)$, describes the rate at which secondary infections are 
expected to be caused by an infected individual, on \emph{average} (see Park 2019
for distinction between individual- and population-level infection kernel). Infection
kernel can be decomposed into two parts: reproduction number, 
$$
\Ro = \int K(s) ds,
$$ 
and the generation-interval distribution, 
$$
g(\tau) = \frac{K(\tau)}{\Ro},
$$
which represents the expected time distribution of the secondary cases. The 
generation-interval distribution plays an important role in liking speed and
strength of an epidemic.

Renewal equation:
\begin{equation}
i(t) = S(t) \int K(s) i(t-s) ds
\end{equation}

Euler-Lotka:
\begin{equation}
1/\RR = \int g(s) \exp(-rs) ds
\end{equation}

\subsection{Control strategies}

\begin{itemize}
	\item $L(\tau)$ represents intervention over the course of infection
	\item Then, $\hat K(\tau) = K(\tau)/L(\tau)$
	\item Define intervention strength:
	$$
	\theta = 1/\langle 1/L(\tau) \rangle_{g(\tau)},
	$$
	harmonic mean of $L(\tau)$ weighted by generation-interval distribution; outbreak controlled if $\theta > \mathcal R$ (not affected by $\RR$)
	\item Define intervention speed ($\phi = r - \hat r$):
	$$
	1 = \langle \frac{\exp(\phi \tau) }{L(\tau)} \rangle_{b(\tau)}
	$$
	sort of a mean associated with $L(\tau)$; outbreak controlled if $\phi > r$ (not affected by $r$).
\end{itemize}

\subsection{HIV example? Maybe more?}

\begin{itemize}
	\item Read ``Factors that make an infectious disease outbreak controllable''
	\item Can active testing and treatment stop the epidemic?
	\item Explain different scenarios? assumptions?
	\item Early vs late transmission (change proportion of early transmission)
	\item Compare required strength and speed of intervention required to control the disease
	\item Timing of target intervention (Fraser works on this stuff and did a clinical trial on this; unpublished)
\end{itemize}

\section{Results}

\begin{itemize}
	\item Early transmission matters for condom
	\item Early transmission doesn't matter so much for target and treat
	\item Think about speed vs strength axis
\end{itemize}

\section{Discussion}


\begin{itemize}
	\item Talk about generation time and reinterpret Eaton and Hallet?
	\item Short generation time -> teating and treating will help less but lower epidemic strength
	\item High generation time -> high proportion prevented but higher R.
\end{itemize}

\begin{itemize}
	\item r and R have more in common than we think
	\item Why test and treat predictions are robust to assumptions about transmission
\end{itemize}



\end{document}
