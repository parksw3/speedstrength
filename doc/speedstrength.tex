\documentclass{article}\usepackage[]{graphicx}\usepackage[]{color}

\title{Speed and strength of epidemic intervention}
\author{Jonathan Dushoff, Sang Woo Park}

\usepackage{tabularx}

%% \usepackage{amsmath}
\usepackage{natbib}
\usepackage{hyperref}
\bibliographystyle{chicago}
\date{\today}

\usepackage{bm}

\usepackage{afterpage}
\usepackage{pdflscape}

\newcommand{\etal}{\textit{et al.}}

\newcommand{\comment}[3]{\textcolor{#1}{\textbf{[#2: }\textit{#3}\textbf{]}}}
\newcommand{\jd}[1]{\comment{cyan}{BMB}{#1}}
\newcommand{\swp}[1]{\comment{magenta}{SWP}{#1}}

\newcommand{\Rx}[1]{\ensuremath{{\mathcal R}_{#1}}} 
\newcommand{\Ro}{\Rx{0}}
\newcommand{\RR}{\ensuremath{{\mathcal R}}}
\newcommand{\Rhat}{\ensuremath{{\hat\RR}}}
\newcommand{\tsub}[2]{#1_{{\textrm{\tiny #2}}}}

\newcommand{\figref}[1]{Fig.~\ref{fig:#1}}
\newcommand{\figlab}[1]{\label{fig:#1}}
\newcommand{\eqref}[1]{(\ref{eq:#1})}
\newcommand{\eqlab}[1]{\label{eq:#1}}
\begin{document}

\maketitle

\section{Introduction}

An epidemic can be characterized by its \emph{speed} and \emph{strength}.
The speed of an epidemic, often characterized by the exponential growth rate $r$, measures how \emph{fast} an epidemic grows, at the population level. 
The strength of an epidemic, often characterized by the reproductive number $\RR$, measures how many new cases are caused by a typical \emph{individual} case.
Knowing the speed and strength of an epidemic allows predictions about the course of the epidemic and the effectiveness of intervention strategies.

Much research has prioritized estimates of the reproductive number, because it has a threshold value, $\RR=1$ that determines whether a disease can invade, the level of equilibrium, and the effectiveness of control efforts. 
The insight that a case must on average cause at least one new case under good conditions for a disease persist goes back $>100$ years \cite{Ross}, and the idea of averaging by defining a `typical' case was formalized in the 19x0s \cite{Heesterbeek}.
$\RR$ is also of interest because it provides a \emph{prima facie} prediction about the total \emph{size} of an epidemic. 

Like epidemic strength, epidemic speed measures how disease spreads in the population. Here, we generalize the idea of a threshold for successful intervention by measuring an intervention's strength on the same scale as the reproductive number. We then show that there is a duality between the threshold $\RR=1$ and a corresponding minimal intervention strength, and the threshold $r=0$ and a corresponding minimal intervention speed. We argue that the historical primacy of $\RR$ over $r$ is partly artificial, and discuss cases where strength provides the better framing for practical disease questions and cases where speed does.

\section{Methods}

\subsection{Epidemic model}

We model disease incidence using the renewal-equation framework \cite{SeeChamp?}. 
This is a simple, flexible framework that can cover a wide range model structures.
In our model, disease incidence at time $t$ is given by:
\begin{equation}
i(t) = \int K(\tau, t) i(t-\tau) d\tau.
\end{equation}
Here, $K$ is the infection kernel describing how infectious we expect an individual infected $\tau$ time units ago to be in the population.
In general, $K$ will depend on population characteristics that may change through time -- notably, the proportion of the population susceptible.
Since we are interested in invasion and control, we will generally neglect changes in $K$ through time, thus we will assume $K\equiv K(\tau)$. 
This means we are neglecting changes in susceptible proportion through time.

\subsection{Strength-based decomposition}

Assuming that $K$ doesn't change with time, we write:
\begin{equation}
	K(\tau) = \RR g(\tau),
	\eqlab{speed}
\end{equation}
where $g(\tau)$ is the ``intrinsic'' generation-interval distribution (giving the relative infectiousness of the average individual as a function of time since infection) \cite{Champ}. Since $g$ is a distribution, it integrates to 1, and $\RR$ is thus the integral of $K$.

Imagine a control measure that proportionally reduces $K$, so that the post-intervention kernel $\hat K = K/\theta$.  The factorization \eqref{speed} show that the reduction needed to prevent invasion (or to eliminate disease)  is exactly \RR. We call $\theta$ the ``strength'' of the intervention; transmission is interrupted when the strength of the intervention is larger than the strength of spread.

We generalize this idea by allowing an intervention strategy to reduce $K$ by different proportions over the course of an individual infection. We write the post-intervention kernel:
\begin{equation}
	\hat K(\tau) = K(\tau)/L(\tau), 
\end{equation}
where $L(\tau)$ is the average proportional reduction for an individual infected time $\tau$ ago.

We define the strength of the intervention $L$ to be $\theta = \RR/\Rhat$. It is then straightforward to show that $\theta$ is the harmonic mean of $L(\tau)$ weighted by generation-interval distribution.

\begin{equation}
	\theta = 1/\langle 1/L(\tau) \rangle_{g(\tau)},
\end{equation}

\subsection{Speed-based decomposition}

By analogy with the strength-based factorization \eqref{speed}, we can rewrite the Euler-Lotka equation \eqref{euler} as a speed-based factorization:

\begin{equation}
K(\tau) = b(\tau)\exp(r\tau)
\end{equation}

Like $g$, $b$ is a distribution: in this case the initial backward generation interval, which gives the distribution of realized generation times when the disease spreads exponentially.

Infection kernel, $K(\tau)$, describes the rate at which secondary infections are 
expected to be caused by an infected individual, on \emph{average} (see Park 2019
for distinction between individual- and population-level infection kernel). Infection
kernel can be decomposed into two parts: reproduction number, 
$$
\Ro = \int K(s) ds,
$$ 
and the generation-interval distribution, 
$$
$$
which represents the expected time distribution of the secondary cases. The 
generation-interval distribution plays an important role in linking speed and
strength of an epidemic.


	Define intervention speed ($\phi = r - \hat r$):
	$$
	1 = \left\langle \frac{\exp(\phi \tau) }{L(\tau)} \right\rangle_{b(\tau)}
	$$
	sort of a mean associated with $L(\tau)$; outbreak controlled if $\phi > r$ (not affected by $r$).


\section{Examples}

\subsection{Test and treatment - HIV}

\begin{itemize}
	\item Can active testing and treatment stop the epidemic?
	\item Explain different scenarios? assumptions?
	\item Early vs late transmission (change proportion of early transmission)
	\item Compare required strength and speed of intervention required to control the disease
	\item Early transmission matters for condom
	\item Early transmission doesn't matter so much for target and treat
	\item Think about speed vs strength axis
\end{itemize}

\subsection{Asymptomatic infection}

\begin{itemize}
	\item See ``Factors that make an infectious disease outbreak controllable'' by Fraser et al.
	\item Our framework generalizes their work
	\item Timing of target intervention (Fraser works on this stuff and did a clinical trial on this; unpublished)
\end{itemize}

\subsection{Maybe one more example}

\section{Discussion}


\begin{itemize}
	\item Talk about generation time and reinterpret Eaton and Hallet?
	\item Short generation time -> teating and treating will help less but lower epidemic strength
	\item High generation time -> high proportion prevented but higher R.
\end{itemize}

\begin{itemize}
	\item r and R have more in common than we think
	\item Why test and treat predictions are robust to assumptions about transmission
\end{itemize}



\end{document}
